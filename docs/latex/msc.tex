\documentclass[a4paper,12pt]{report}

% Page layout
\usepackage[left=2.5cm,right=2.5cm,top=2.5cm,bottom=2.5cm]{geometry}

% Font and text
\usepackage[afrikaans,english]{babel}
\usepackage{microtype}
\usepackage{setspace}
\usepackage{lmodern}
\usepackage{siunitx}
\usepackage{tcolorbox}
% \usepackage[scaled=.96]{XCharter}
\usepackage[scaled=.96]{caladea}  % font required by Stellenbosch University
\usepackage[scaled=.96,lf]{carlito}  % the scaling makes the math the same height
\renewcommand*\oldstylenums[1]{\carlitoOsF #1}
\newcommand{\myemph}[1]{{\sffamily\bfseries#1}}
\sloppy
\onehalfspacing

% Headings
\usepackage[raggedright,sf,bf]{titlesec}
\usepackage[margin=\the\parindent,small,bf,sf]{caption}
\titlelabel{\thetitle.\ }
\titleformat{\chapter}[display]{\huge\bfseries\sffamily}{\chaptertitlename\ \thechapter}{15pt}{\Huge \raggedright}
\titlespacing*{\chapter}{0pt}{0pt}{40pt}  % remove spacing before chapter headings
\makeatletter
\let\originall@chapter\l@chapter
\def\l@chapter#1#2{\originall@chapter{{\sffamily #1}}{#2}}
\makeatother

%% Alternative headings using small-caps (comment out the top section)
%\usepackage[raggedright,bf]{titlesec}
%\usepackage[margin=\the\parindent,small,bf]{caption}
%\titlelabel{\thetitle.\ }
%\titleformat{\chapter}[display]{\huge\scshape}{\chaptertitlename\ \thechapter}{15pt}{\Huge \raggedright}
%\titlespacing*{\chapter}{0pt}{0pt}{40pt}  % remove spacing before chapter headings

% Table of contents
\let \savenumberline \numberline
\def \numberline#1{\savenumberline{#1.}}

% Figures
\usepackage{graphicx}
\usepackage{tikz}
\usepackage{pdfpages}
\usepackage{subcaption}
\setlength{\abovecaptionskip}{7.5pt}  % spacing above and below captions
\newcommand*{\WaterMark}[2][0.2\paperwidth]{\AddToShipoutPicture*{\AtTextCenter{\parbox[c]{0pt}{\makebox[0pt][c]{\includegraphics[width=#1]{#2}}}}}}

% Mathematics
\usepackage[cmex10]{amsmath}
\usepackage{amssymb}
\usepackage{cancel}
\DeclareMathOperator*{\argmax}{arg\,max}
\newcommand{\T}{^\top}
\newcommand{\tr}{\textrm{tr}}
\renewcommand{\vec}[1]{\boldsymbol{\mathbf{#1}}}
\newcommand{\defeq}{\triangleq}

% Tables
\usepackage{booktabs}
\usepackage{tabularx}
\usepackage{multirow}
\newcommand{\mytable}{
    \centering
    \small
    \renewcommand{\arraystretch}{1.2}
    }
\renewcommand{\tabularxcolumn}[1]{m{#1}}
\newcolumntype{C}{>{\centering\arraybackslash}X}
\newcolumntype{L}{>{\raggedright\arraybackslash}X}

% Header and footer
\usepackage{fancyhdr}
\pagestyle{fancy}
\fancyhf{}
\renewcommand{\sectionmark}[1]{\markright{\normalsize \thesection.\ #1}}
\fancyhead[C]{\nouppercase{\textit{\rightmark}}}
\fancyhead[RO]{\thepage}
\fancyhead[LE]{\thepage}  % double-sided printing
\fancyfoot{}
\setlength\headheight{14.5pt}
\renewcommand{\headrulewidth}{0pt}
\fancypagestyle{plain}{\fancyhead{}
                       \renewcommand{\headrulewidth}{0pt}
                       \fancyfoot[C]{\thepage}}

% Pseudo-code
\usepackage{algorithm}  % should go before \usepackage{hyperref}

% Table of contents and hyperlinks
\usepackage{hyperref}
\hypersetup{colorlinks=true,linktoc=all,citecolor=black,linkcolor=black}
\usepackage[nottoc]{tocbibind}

% Pseudo-code
\usepackage{algpseudocode}  % should go after \usepackage{hyperref}
\renewcommand{\thealgorithm}{\arabic{chapter}.\arabic{algorithm}} 
\captionsetup[algorithm]{labelfont={bf,sf},font=small,labelsep=colon}

% Bibliography
\usepackage{cite}  % automatically reorder inline citations
\bibliographystyle{IEEEtran}

% Fix titlesec issue
\usepackage{etoolbox}
\makeatletter
\patchcmd{\ttlh@hang}{\parindent\z@}{\parindent\z@\leavevmode}{}{}
\patchcmd{\ttlh@hang}{\noindent}{}{}{}
\makeatother

% Custom Packages
\usepackage{marginnote}

\newcommand{\note}[1]{%
    \marginnote{\textcolor{red}{#1}}%
}
\usepackage{parskip}

\begin{document}

% Front matter
% \graphicspath{{frontmatter/fig/}}
\pagenumbering{Alph}

\begin{titlepage}
    \begin{tikzpicture}[remember picture, overlay]
        \node [opacity=1.0, anchor=center] at (current page.center) 
        {\includegraphics[width=\paperwidth]{stb-thesis-frntp.pdf}};
    \end{tikzpicture}

	\begin{center}			
		\vfill
        \vfill
        \vfill
		{\bfseries \huge A Contrastive Approach to Weight Space Learning \par}
		\vfill
        {\large by \\[5pt]}
		{\Large {\Large D.J. Swanevelder} \par}
		\vfill
		\vfill

		{\large MSc Machine Learning/ Artificial Intelligence \par
		 Department of Applied Mathematics \par Faculty of Science \par Stellenbosch University \par}
		
		\vfill
		
		{\large {Supervisor}: Ruan van der Merwe}\par
		\vfill
		{\large November 2025}
        \vfill
	\end{center}
\end{titlepage}
 % include
\pagenumbering{roman} 
% \thispagestyle{plain}

\chapter*{Declaration}
\addcontentsline{toc}{chapter}{Declaration}

% \vspace{20pt}

% Masters (Research)
By submitting this project electronically, I declare that the entirety of the work contained
therein is my own, original work, that I am the sole author thereof (save to the extent explicitly
otherwise stated), that reproduction and publication thereof by Stellenbosch University will
not infringe any third party rights and that I have not previously in its entirety or in part
submitted it for obtaining any qualification.

In the preparation of this project, artificial intelligence (AI) tools were utilised solely as supportive instruments to enhance the quality and efficiency of the research process. The use of AI was limited to improving the clarity, structure, and grammatical quality of the written content. All technical content, experimental design, analysis, and conclusions presented are entirely the result of my own work, understanding, and interpretation. Any text generated through AI assistance were critically reviewed, revised, and restructured to ensure full alignment with my personal voice, intended meaning, and academic integrity.

With regard to coding, AI tools were employed to expedite routine programming tasks such as generating boilerplate code, refactoring existing code for improved readability, and identifying potential bugs. All AI-assisted code was carefully validated, modified where necessary, and verified to behave as intended. The use of AI in this context served only to improve workflow efficiency and did not influence the originality or integrity of the implemented methods or results.

\vspace{35pt}
% \vspace{1cm}
\noindent
\begin{minipage}{.5\textwidth}
    \noindent
    \phantom{Date:}~\hfill\makebox[0pt][c]{November 2025}\hfill\mbox{}\\[-.5\baselineskip]
    Date:~ \dotfill\mbox{}\par
\end{minipage}

\vspace{35pt}

\vfill

% \begin{center}
    Copyright © 2025 Stellenbosch University\par
    All rights reserved
% \end{center}

\vspace{35pt}


% \chapter*{Abstract}
\addcontentsline{toc}{chapter}{Abstract}
\makeatletter\@mkboth{}{Abstract}\makeatother
% TODO {abstract} 
Neural networks have become a cornerstone of modern artificial intelligence, but challenges remain in understanding their internal representations and efficiently generating new models. Weight space learning offers a framework for addressing these challenges by modelling the structure of neural network parameters and their relationship to datasets and performance. In this work, we propose a contrastive learning approach that jointly embeds neural network weights, the datasets they are trained on, and their resulting performance metrics into a unified latent space. Separate encoders for dataset and weight representations, combined with a binned results embedding, are trained with a contrastive objective to align these heterogeneous modalities. This unified embedding space enables interpretability—by revealing relationships between datasets and model behaviour—and conditional model sampling, approximating the joint distribution \(P(W \mid \mathcal{D}, R)\). Experimental results highlight limitations in current weight encoding strategies, particularly linear PCA compression, which constrain generalisation. Nonetheless, the framework demonstrates the potential of contrastively aligning heterogeneous modalities, providing a foundation for future work aimed at improving embedding quality, expanding model diversity, and enhancing conditional model generation.

% TODO {abstract} 

 % include
% \chapter*{Acknowledgements}
% \addcontentsline{toc}{chapter}{Acknowledgements}
\makeatletter\@mkboth{}{Acknowledgements}\makeatother
% TODO {Acknowledgements}
    I would like to thank my cat, Muffin. I also would like to thank the inventor of the incubator; without him/her, I would not be here. Finally, I would like to thank Dr Herman Kamper for this amazing report template.
% TODO {Acknowledgements}

 % include
% \tableofcontents % include
%\listoffigures
%\listoftables
% \input{frontmatter/nomenclature}
\newpage
\pagenumbering{arabic}

% Contents
\graphicspath{{fig/}}

\chapter{Introduction}
\label{chap:introduction}

\begin{quote}
    "We don't tell [computers] what to do, we give them examples... The problem is, sometimes we don't understand how it figured it out."\\ % <--- FORCED NEW LINE HERE
    \vspace{0.5em} % Adds a slightly larger vertical space after the quote text
    \hfill -- Jeff Dean, Head of Google AI \cite{Dean2017BlackBox}
\end{quote}

The prevalence of neural networks (NNs) has established them as a foundational technology in modern artificial intelligence. However, as their use has expanded, so too has attention to their inherent mechanistic limitations. Two major drawbacks of NNs are their lack of explainability—the “black-box” effect—and the substantial computational cost of training. With platforms like Hugging Face and GitHub hosting over one million pre-trained models \cite{huggingface2024review}, interest in addressing these limitations has intensified. Motivated by this vast availability of models, a novel research direction—\textbf{weight space learning}—has emerged as a promising approach to better understand these limitations.

Formally, the weight space of a neural network refers to the set of all possible configurations of its parameters \( W \in \mathbb{R}^n \), where \( n \) is the total number of trainable weights. Each point in this space represents a unique model with a distinct mapping from inputs to outputs. Weight space learning thus concerns learning representations or distributions over this space, capturing how variations in \( W \) relate to model behaviour and how variations in model training influence \( W \). The field generally considers two types of tasks: discriminative and generative.

In discriminative applications, models use the weights of pre-trained networks—often collected into a model zoo \cite{schurholt2022modelzoosdatasetdiverse}—as input to predict meta-information about the original models \cite{unterthiner2021predictingneuralnetworkaccuracy}. The quality of a weight space representation is typically evaluated by the performance of a simple multi-layer perceptron (MLP) in predicting such meta-information, conditioned only on the model's weight embedding.

Common meta-information metrics include the model's final performance and its generalisation gap (the difference between training and validation loss). \cite{salama2024datasetsizerecoverylora} demonstrated that weight embeddings can encode key training characteristics, such as recovering the size of the dataset used for training. These discriminative tasks serve both to validate the quality of the derived weight representations and to provide practical predictive value.

In generative applications, researchers aim to model the underlying distribution of neural network weights \( W \), conditioned on additional information or reference models, \( P(W \mid \dots) \). Sampling from this distribution enables the generation of entirely new model weights.

\cite{schurholt2022hyperrepresentationsgenerativemodelssampling} used an autoencoder with a bottleneck layer to generate hyper-representations of multiple model zoos. Building on this idea, \cite{pmlr-v235-schurholt24a} introduced the Sequential Autoencoder for Neural Embeddings (SANE), which improved scalability and enabled work on much larger models.

To model \( P(W \mid \mathcal{D}) \), \cite{bedionita2025instructionguidedautoregressiveneuralnetwork} employed a Vector Quantised Variational Autoencoder (VQ-VAE), which incorporates dataset information when learning latent weight representations. Similarly, \cite{meynent2025structureenoughleveragingbehavior} modelled \( P(W \mid R) \) by incorporating behavioural differences between reconstructed and original models into the embedding learning process.

While existing weight space learning methods have made significant progress, they typically address isolated aspects of the learning process. Current approaches model either \( P(W \mid \mathcal{D}) \) or \( P(W \mid R) \), but rarely both simultaneously. This is a key limitation: in practice, a model's weights are influenced by both the data it was trained on and the results it achieved. Understanding the joint relationship \( P(W \mid \mathcal{D}, R) \) is essential for explaining model behaviour and for generating models with desired characteristics.

Contrastive learning has emerged as a powerful paradigm for learning unified representations across modalities, as demonstrated by models such as CLIP \cite{radford2021learningtransferablevisualmodels}, which bridge vision and language. A contrastive objective pulls related samples closer in representation space while pushing unrelated samples apart, forming a meaningful joint embedding space for heterogeneous data types. This property makes contrastive learning particularly suitable for modelling the complex distribution \( P(W \mid \mathcal{D}, R) \) --- encompassing visual datasets, high-dimensional weight tensors, and performance metrics --- without requiring a shared native representation.

\begin{figure}[!t]
    \centering
    \includegraphics[width=0.75\linewidth]{pipeline.png}
    \caption[The full pipeline of embedding a dataset, model weights and results in a shared embedding space.]{The full pipeline of embedding a dataset, model weights and results in a shared embedding space. }
    \label{fig:pipeline}
\end{figure}

In this report, we develop a contrastive learning framework to create a unified embedding space that jointly represents neural network weights \( W \), the datasets they were trained on \( \mathcal{D} \), and their resulting performance characteristics \( R \). Specifically, we construct two separate encoders—one for dataset embeddings using pre-trained CLIP features, and another for weight embeddings using an autoencoder architecture—alongside a binned result embedding table. These encoders are trained using the contrastive objective NT-Xent \cite{agren2022ntxentlossupperbound}, which encourages related triplets \( (\mathcal{D}, W, R) \) to be close in the shared latent space. Figure \ref{fig:pipeline} depicts a high-level view of the full embedding pipeline.
\newpage

Our central \textbf{hypothesis} is that this unified representation space will enable two key capabilities:

\begin{itemize}
    \item \textbf{Interpretability and Analysis:} Examining geometric relationships within the shared embedding space can reveal how dataset characteristics shape learned weights and model behaviour.
    \item \textbf{Conditional Model Sampling:} The learned distribution enables sampling of model weights conditioned on both dataset properties and target performance metrics--approximating \( P(W \mid \mathcal{D}, R) \).
\end{itemize}

\textbf{Primary Goal}: The primary goal of this report is to demonstrate the viability of our methodology through the success of \textbf{Conditional Model Sampling.} While our hypothesis includes both Interpretability and Conditional Model Sampling, success in the latter strongly suggests a meaningful foundation for the former. Specifically, the ability to accurately sample weights that achieve target performance metrics conditioned on specific dataset ($\mathcal{D}$) and result ($R$) targets confirms that the unified embedding space has meaningfully captured the joint influence of $\mathcal{D}$ and $R$ on the model weights $W$. We will therefore focus on generating meaningful representations, measuring their quality, and critically, performing various forms of conditional model sampling to determine the viability of this specific methodology for generative weight space representations.

The remainder of this report is structured as follows. In Chapter \ref{chap:background}, we discuss the necessary background on weight space learning and contrastive learning. Chapter \ref{chap:method} details the methodology, covering the encoders for $\mathcal{D}$, $W$, and $R$, and the shared contrastive method. Chapter \ref{chap:results} presents our experimental results, focusing on the quality of the learned representations and the core objective of conditional model sampling. Finally, Chapter \ref{chap:conclusion} concludes the report and suggests directions for future work.
 % include
\chapter{Background}
\label{chap:background}

% TODO {Weights Space Litarature summary}
    
% TODO {Weights Space Litarature summary}
\section{Weight Space Learning}

Weight space learning is a field within machine learning that focuses on understanding and leveraging the structure of the neural network weight space. The central aim is to model how network parameters are shaped by data, architecture, and training dynamics, and to capture these relationships within a learnable representation. 

At its core, weight space learning seeks to construct \textit{meta-models}—models that learn from other models. Unlike standard machine learning models that capture patterns in data, meta-models capture patterns in the \textit{weights} of networks trained on that data. In this way, the goal shifts from learning a direct input--output mapping to learning the structure that governs how such mappings are formed.

Due to the enormous scale and dimensionality of modern neural networks~\cite{}, it is typically infeasible to operate directly on raw model weights. Furthermore, redundancy is well known to exist in neural networks; smaller architectures can often achieve comparable performance to larger ones~\cite{}. These challenges motivate one of the central subproblems of weight space learning: the discovery of low-dimensional representations of weight space.

A \textit{latent representation} of weight space provides a compact and structured encoding of a model's parameters. The transformations—linear or non-linear—that map weights into this latent space are learned to preserve the essential information required to reconstruct or analyse the original weights. Among the many dimensionality reduction techniques available, a useful distinction can be made between \textit{reversible} and \textit{non-reversible} methods.

Reversibility is of particular importance in weight space learning. While encoding weights into a latent representation (real $\rightarrow$ latent) is informative, the ability to reconstruct the original weights (real $\rightarrow$ latent $\rightarrow$ real) is far more valuable. This reversibility enables the synthesis of entirely new weight configurations, supporting generative applications such as zero-shot model creation and performance-guided model generation. Consequently, reversible latent representation methods are the most prevalent within weight space learning, especially for encoding and decoding neural network weights.

This chapter proceeds as follows. Section~\ref{sec:pca} introduces Principal Component Analysis (PCA), a linear and probabilistic approach that provides a simple yet effective reversible dimensionality reduction method. Section~\ref{sec:autoencoders} discusses reversible, non-linear approaches, focusing on autoregressive encoder architectures and presenting the Sequential Autoencoder for Neural Embeddings (SANE)~\cite{}. Finally, Section~\ref{sec:contrastive} explores a non-reversible, modality-heterogeneous technique, contrastive learning, including a description of the NT-Xent loss and its implementation in CLIP~\cite{}.

\section{Principal Component Analysis (PCA)}
\label{sec:pca}

Principal Component Analysis (PCA) is a linear dimensionality reduction technique used to represent high-dimensional data in a lower-dimensional form while retaining as much variance as possible. It provides a compact latent representation that captures the most informative directions of variation in the data. This section outlines the motivation behind PCA, its mathematical formulation, and its relevance and limitations as a latent representation method.

\vspace{0.5em}
\noindent
\textbf{Problem Setup and Intuition.}
Consider a dataset \( X \in \mathbb{R}^{n \times d} \) with \( n \) samples and \( d \) features, centred such that each feature has zero mean. The goal of PCA is to find a new coordinate system whose axes are linear transformations of the original dimensions, ordered by the amount of variance they capture. Orthogonality between these axes ensures that each captures unique, non-redundant information about the data.

Intuitively, PCA identifies the directions that best describe the “shape” or spread of the data cloud in feature space. Projecting the data onto the top \( k \) directions provides a compressed representation that preserves most of the information while discarding redundancy.

\vspace{0.5em}
\noindent
\textbf{Mathematical Formulation.}
PCA seeks a linear projection matrix \( W \in \mathbb{R}^{d \times k} \) that maps the data to a lower-dimensional space:
\[
Z = XW,
\]
where \( Z \in \mathbb{R}^{n \times k} \) represents the latent representation.  
The sample covariance matrix of \( X \) is
\[
\Sigma = \frac{1}{n} X^\top X,
\]
and the total variance captured by the projection is
\[
\text{Var}(Z) = \text{Tr}(W^\top \Sigma W),
\]
where the trace operator \( \text{Tr}(\cdot) \) sums the variances along all projected directions.  

PCA thus maximises the variance of the projected data:
\[
\max_{W} \text{Tr}(W^\top \Sigma W)
\quad \text{subject to} \quad W^\top W = I_k.
\]
The constraint enforces orthonormality among the new axes so that each captures distinct variance. Solving this optimisation leads to the eigenvalue problem:
\[
\Sigma W = W \Lambda,
\]
where the columns of \( W \) are the eigenvectors of \( \Sigma \), and the diagonal entries of \( \Lambda \) are the corresponding eigenvalues that quantify the variance explained by each principal component. The top \( k \) eigenvectors define the optimal projection directions.

\vspace{0.5em}
\noindent
\textbf{Latent Representation and Limitations.}
The resulting embedding,

\begin{equation}
    Z = X W_k
  \label{eq:pca}
\end{equation}

is the \textit{latent representation} of the data, capturing the dominant linear structure of the dataset.  
PCA provides a reversible mapping that probabilistically preserves the greatest possible amount of variance under a linear projection. 

However, its linear nature limits its ability to capture nonlinear relationships when the data lie on curved manifolds. Furthermore, PCA does not adapt to unseen data that deviate from the original subspace—it yields a fixed, non-learnable transformation.

\vspace{0.5em}
\noindent
\textbf{Incremental PCA.}
In practice [cite], the covariance matrix \( \Sigma \) can be prohibitively large for high-dimensional data, such as neural network weights. Incremental PCA (IPCA) addresses this by processing data in smaller batches and updating the principal components iteratively. Rather than recomputing the full covariance, it approximates it using partial updates and truncated singular value decompositions (SVD).  
This approach makes IPCA suitable for large-scale or streaming datasets while maintaining results comparable to standard PCA. However, it remains an approximation and inherits PCA’s linear and non-generalisable nature.

\section{Autoencoders}
\label{sec:autoencoders}

Autoencoders are neural network architectures designed for unsupervised representation learning. Their goal is to learn a compressed, informative latent representation of input data by training the network to reconstruct the original input after passing it through a lower-dimensional bottleneck. An Autoencoder's architecture is characterized by a layer, with less nodes than the dimension of the input data, then expanding to the size of the original data. 

\vspace{0.5em}
\noindent
\textbf{Architecture and Operation.}
An autoencoder consists of two primary components: an \textit{encoder} and a \textit{decoder}. The encoder, parameterised by weights $\theta_e$, maps an input $x \in \mathbb{R}^d$ to a latent representation $z \in \mathbb{R}^k$ through a sequence of nonlinear transformations:
\begin{equation}
z = f_\text{enc}(x; \theta_e).
\label{eq:encoder}
\end{equation}
The decoder, parameterised by $\theta_d$, reconstructs the input from this latent representation:
\begin{equation}
\hat{x} = f_\text{dec}(z; \theta_d).
\label{eq:decoder}
\end{equation}
Together, the encoder and decoder form a composite function:
\begin{equation}
\hat{x} = f_\text{dec}(f_\text{enc}(x)).
\label{eq:composite}
\end{equation}
The bottleneck layer (latent space) enforces a compression constraint, ensuring the model retains only the most salient features necessary for accurate reconstruction.

\vspace{0.5em}
\noindent
\textbf{Training Objective.}
The network is trained to minimise the reconstruction error between the input $x$ and its reconstruction $\hat{x}$. The most common loss function is the Mean Squared Error (MSE):
\begin{equation}
\mathcal{L} = \frac{1}{n} \sum_{i=1}^{n} \| x_i - \hat{x}_i \|^2.
\label{eq:ae_mse_loss}
\end{equation}
This loss is automatically differentiated and optimised through backpropagation using gradient descent. By minimising this objective, the autoencoder learns weight configurations that best compress and reconstruct the data distribution.

The objective can be extended to include additional terms depending on the desired properties of the latent space. For example, a \textit{contrastive loss} component can be incorporated to encourage semantic separation between latent representations, improving feature discriminability. Likewise, \textit{regularisation} terms—such as $L_1$ or $L_2$ penalties, or sparsity constraints—can be added to improve generalisation and prevent overfitting by controlling the magnitude or distribution of learned weights.


\vspace{0.5em}
\noindent
\textbf{Latent Representation and Flexibility.}
The latent space $z$ provides a nonlinear embedding that captures underlying structure within the data. Once trained, the encoder can be used independently for dimensionality reduction or feature extraction, while the decoder can serve as a generative mapping from the latent space back to the input domain.

Autoencoders are highly flexible in architecture — they can be shallow for simple data or deep to capture hierarchical structure. Nonlinear activation functions such as ReLU, tanh, or sigmoid increase expressivity, allowing the model to represent complex, nonlinear manifolds within the data distribution.

\subsection{Autoencoder for Neural Embeddings}
Applying the concept of an autoencoder to neural network weights has recently gained attention as a means to learn structured, low-dimensional representations of model parameters. Instead of encoding raw input data, the autoencoder learns to encode and reconstruct the weights of pre-trained neural networks, effectively embedding each model into a latent space that captures structural and functional similarities between models. This approach was explored in \cite{NEURIPS2022_b2c4b7d3}, where the goal was to build a continuous and interpretable representation space of neural weights.

\vspace{0.5em}
\noindent
\textbf{Training Objective.}
Unlike conventional autoencoders that optimise a single reconstruction loss, the neural weight autoencoder is trained with a multi-objective loss function that combines reconstruction and contrastive terms:
\begin{equation}
\mathcal{L} = \beta \mathcal{L}_{\text{MSE}} + (1 - \beta) \mathcal{L}_{\text{c}},
\label{eq:multi_loss}
\end{equation}
where $\mathcal{L}_{\text{MSE}}$ represents the weight reconstruction loss and $\mathcal{L}_{\text{c}}$ is a contrastive loss.  
The reconstruction term encourages the decoder to accurately reproduce the original model weights, expressed layer-wise as:
\begin{equation}
\mathcal{L}_{\text{MSE}} = \frac{1}{MN} \sum_{i=1}^{M} \sum_{l=1}^{L} \| \hat{w}_i^{(l)} - w_i^{(l)} \|_2^2,
\label{eq:weight_recon_loss}
\end{equation}
where $w_i^{(l)}$ and $\hat{w}_i^{(l)}$ are the original and reconstructed weights for the $l$-th layer of the $i$-th model in the collection (often called a \textit{model zoo}).  

The contrastive component $\mathcal{L}_{\text{c}}$ introduces additional structure in the latent space by applying data augmentations during training—such as weight permutation (which leverages inherent symmetries in neural networks) and random erasing—to ensure that semantically similar models are mapped closer together while dissimilar ones are pushed apart. In this work, the contrastive term follows the \textit{NXTne} loss formulation (see Section~\ref{sec:nxtne_loss}), which provides a more stable and expressive contrastive objective for modelling relationships in weight space.


This combination of reconstruction and contrastive learning encourages the encoder to capture not only numerical similarity in the weights but also functional and architectural relationships between models. The result is a structured latent space in which proximity correlates with similarity in performance or behaviour.

\vspace{0.5em}
\noindent
\textbf{Interpretation and Utility.}
The learned latent space enables both discriminative and generative applications. The encoder can be used to extract informative embeddings that summarise a model’s functional behaviour, useful for clustering, transfer learning, or model retrieval. Meanwhile, the decoder can generate entirely new sets of weights from latent vectors—supporting generative applications such as zero-shot model synthesis or interpolation between models. This dual capability makes the autoencoder framework particularly appealing for weight space learning, as it provides both interpretability and controllability.

\vspace{1em}
\noindent
\textbf{Sequential Autoencoder for Neural Embeddings.}
A major challenge in encoding neural network weights lies in the enormous number of parameters, which quickly becomes infeasible to process directly. \cite{pmlr-v235-schurholt24a} introduced the Sequential Autoencoder for Neural Embeddings (SANE), which addresses this scalability problem by tokenising model weights rather than treating the entire parameter set as a single input.  

Instead of encoding full weight tensors, SANE partitions them into smaller, semantically meaningful \textit{tokens}—for example, by layer or even within layers—and encodes these sequentially. Each token is represented as a point in latent space, and the entire model is represented as a \textit{cloud of latent tokens} rather than a single vector. This design assumes that sufficient information about the model’s global behaviour is preserved through these local token representations, an assumption supported by empirical results in \cite{pmlr-v235-schurholt24a}.

SANE achieves state-of-the-art performance in both discriminative and generative weight-space applications, demonstrating that this token-based decomposition maintains the essential structure of models while dramatically improving scalability. Furthermore, because the approach is sequential, it naturally accommodates heterogeneous architectures—normalising across different layer types and sizes—and allows the generation of new architectures by decoding variable-length token sequences.  

This flexibility marks an important step toward scalable and architecture-agnostic representation learning in weight space.

\section{Contrastive Learning}
\label{sec:contrastive}
The overall goal of contrastive learning is to learn meaningful representations by comparing data points against one another, rather than relying on explicit labels \cite{}. The foundational assumption is that if a model is provided with sufficient examples of similar and dissimilar pairs, it can learn to represent data such that semantically similar samples lie close together in the embedding space, while dissimilar samples are placed farther apart. This framework enables unsupervised or self-supervised learning of latent representations that capture semantic structure purely through relative similarity.

\subsection{NT-Xent Loss}
\label{sec:nxtne_loss}
A widely used objective for contrastive learning is the Normalised Temperature-Scaled Cross Entropy (NT-Xent) loss introduced in \cite{}. This loss formulates the contrastive task as a classification problem over positive and negative pairs within a batch.  
Given a batch of $N$ samples, each with an augmented pair $(x_i, x_i')$, the loss for a positive pair $(i, j)$ is defined as:
\begin{equation}
  \mathcal{L}_{i,j} = -\log \frac{\exp(\text{sim}(z_i, z_j) / \tau)}{\sum_{k=1}^{2N} \mathbb{1}_{[k \neq i]} \exp(\text{sim}(z_i, z_k) / \tau)}
  \label{eq:nt-xent_loss}
\end{equation}
where $\text{sim}(z_i, z_j)$ denotes the cosine similarity between the latent representations $z_i$ and $z_j$, and $\tau$ is a temperature parameter that controls the sharpness of the distribution. The total batch loss is obtained by averaging over all positive pairs.  
This formulation encourages positive pairs (augmentations of the same sample) to have high similarity, while simultaneously pushing apart embeddings from different samples, resulting in well-structured and discriminative latent representations.

\subsection{CLIP}
An important application of contrastive learning using the NT-Xent objective is CLIP (Contrastive Language--Image Pretraining) \cite{}. CLIP jointly trains an image encoder and a text encoder to align visual and textual representations in a shared embedding space.  
During training, each image is paired with a corresponding caption, forming a positive pair, while all other image--text combinations in the batch act as negatives. The model optimises a symmetric contrastive objective, where each image predicts its matching caption and vice versa:
\[
\mathcal{L}_{\text{CLIP}} = \frac{1}{2}(\mathcal{L}_{\text{image-to-text}} + \mathcal{L}_{\text{text-to-image}}).
\]
Through this process, CLIP learns general-purpose visual and linguistic representations that are semantically aligned. Once trained, the model can perform zero-shot classification and other cross-modal tasks by measuring similarity between image and text embeddings without task-specific fine-tuning.
--- % include
\graphicspath{{fig/}}

\chapter{Methodology}
\label{chap:method}

Our hypothesis is that a unified representation space will enable conditional model sampling, allowing us to accurately approximate the conditional probability $P(W \mid \mathcal{D}, R)$. That is, we aim to sample model weights ($W$) conditioned on both specific dataset properties ($\mathcal{D}$) and target performance results ($R$).
    

To achieve this generative capability, our methodology requires a system that can project the three distinct data types—weights, datasets, and results—into a common, meaningful latent space. This mandates three core encoding components followed by a decoding mechanism for generation, the summation of these componets is visualised in Figure ~\ref{fig:pipeline}:


\begin{enumerate}
    \item \textbf{Weight Encoder and Decoder (${W} \rightarrow Z_W \rightarrow \hat{{W}}$):} To compress the high-dimensional weight tensor into a low-dimensional latent vector $Z_W$ and reconstruct the functional weights.
    \item \textbf{Dataset Encoder ($\mathcal{D} \rightarrow Z_{\mathcal{D}}$):} To map the dataset characteristics into a latent vector $Z_D$.
    \item \textbf{Results Encoder ($R \rightarrow Z_R$):} To map scalar performance metrics into a latent vector $Z_R$.
    \item \textbf{Shared Embedding/Alignment:} A mechanism to ensure the resultant latent vectors ($Z_W, Z_{\mathcal{D}}, Z_R$) are meaningfully aligned in a single, unified space $Z$.
\end{enumerate}

In Section~\ref{sec:data_gen}, we introduce a data generation approach that produces a diverse collection of $(\mathcal{D}, \mathcal{W}, \mathcal{R})$ triplets by systematically varying datasets, and optimisation parameters.

Section~\ref{sec:model} presents the design of the full embedding model, including the weight encoder and decoder,dataset encoder , and results embdding. The weight encoder model transforms the high-dimensional weight tensors ($\mathcal{W}$) into compact latent representations.  The dataset encoder maps each dataset ($\mathcal{D}$) into a semantic representation using pretrained CLIP features, while the results encoder transforms validation losses ($\mathcal{R}$) into fixed-size embeddings through a discretised lookup scheme. 

Finally, Section~\ref{sec:shared_enc} integrates these components within a shared latent space.The model learns to align the dataset and result embeddings with their corresponding weight representations, forming a unified space that captures the relationships between data, model parameters, and performance.

\section{Model Zoo Generation}
\label{sec:data_gen}
The construction of the model zoo is fundamentally driven by the need to generate a diverse, reproducible, and analytically valuable dataset of model weights for comprehensive weight-space exploration. 

The model zoo is composed of numerous models trained on tasks sampled from the same underlying distribution. For our study, this distribution is defined by 3-class classification problems drawn from ImageNet \cite{ImageNet_VSS09}, a widely used benchmark known for its diversity and richness of visual features. Restricting tasks to three randomly selected classes strikes a balance between computational tractability and the creation of a large number of distinct training instances. This design ensures that the models learn from high-quality, real-world visual data while providing a flexible and varied task space through the many possible class combinations.

The architecture was consistently set to a ResNet-18 backbone \cite{He2015DeepRL} across the entire zoo population, with only the final fully connected layer adapted for the three-class classification task. This architectural standardisation ensures that any observed differences in weight-space properties are primarily attributable to the systematic variations in training parameters and the task distribution. 

While early weight-space research often focused on smaller models, ResNet-18 introduces a more challenging architecture that remains compatible with modern approaches--enabling direct comparison to recent scalable methods such as \cite{pmlr-v235-schurholt24a}, while exceeding the limitations of earlier techniques like \cite{schurholt2022hyperrepresentationsgenerativemodelssampling}.

To guarantee the model zoo accurately reflects the diversity of the high-dimensional weight space, variation is introduced at the start of each training run. This begins with the random selection of classification classes from ImageNet, with the primary mechanism for systematic variation being the hyperparameter sampling from the distributions detailed in Table 1. Furthermore, the optimizer is randomly selected to be either Adam or Stochastic Gradient Descent (SGD), adding structural diversity by allowing distinct optimization paths.

% \begin{table}[!h]
%     \centering
%     \caption{Model Zoo Hyperparameter Sampling Distributions and Fixed Parameters}
%     \begin{tabularx}{0.8\linewidth}{@{}lX@{}}
%         \toprule
%         Hyperparameter & Distribution / Value \\
%         \midrule
%         Problem Domain & Random subsets of 3 ImageNet classes \\
%         Model Architecture & ResNet-18 (Fixed) \\
%         Optimizer & Randomly selected: Adam or SGD \\
%         Early Epoch Snapshot & Uniform, $U(5, 10)$ \\
%         Maximum Epoch & Truncated Normal, $N(\mu=100, \sigma=15)$ with bounds $[50, 150]$ \\
%         Learning Rate ($\eta$) & $Insert specific LR distribution$ \\
%         \bottomrule
%     \end{tabularx}
%     \label{tbl:model_zoo_params}
% \end{table}


\begin{table}[!h]
    \centering
    \caption{Model Zoo Hyperparameter Sampling Distributions and Fixed Parameters}
    \begin{tabularx}{0.95\linewidth}{@{}lXl@{}}
        \toprule
        Hyperparameter & Distribution / Value & Type \\
        \midrule
        Model Architecture & ResNet-18 & Fixed \\
        Problem Domain & Random subsets of 3 ImageNet classes & Sampled \\
        Optimizer & Randomly selected: Adam or SGD & Sampled \\
        Early Epoch Snapshot & Uniform, $U(5, 10)$ & Sampled \\
        Maximum Epoch & Truncated Normal, $N(100,15)$, clipped to $[50,150]$ & Sampled \\
        \bottomrule
    \end{tabularx}
    \label{tbl:model_zoo_params}
\end{table}


To analyze the geometry of the weight space at different phases of optimization, two distinct weights snapshots are captured per training trajectory: an 'early epoch' checkpoint and a 'max epoch' checkpoint. The early epoch is sampled uniformly, specifically designed to capture models in low-performing, under-trained states. This intentional approach provides crucial data on the structure of the weight space's nascent landscape and the optimization trajectory. Conversely, the maximum training epoch is sampled from a truncated Normal distribution. This range ensures the model has reached a point reflective of a converged, stable local loss minimum, thereby accurately representing the typical, high-performing outcomes of model training. Finally, the total sample size, $1000$, is set to the maximum number computationally feasible given resource constraints. This practical constraint ensures the generation of a statistically representative number of unique training instances necessary for robust analysis of the weight space.

\section{Model}
\label{sec:model}
\subsection{Weight Encoder}
\label{subsec:weight_enc}
The Weight Encoder is implemented to learn a compact, functional latent representation for neural network parameters. The objective is to train an autoencoder that encodes the parameters of each model and reconstructs them with minimal deviation from the originals. This corresponds to minimising the reconstruction loss, $\mathcal{L}_{\text{MSE}}$, as defined in Equation~\ref{eq:weight_recon_loss}. To ensure generalisation, the model is training using on the training set $\mathcal{M}_{\text{train}}$, and performance is validated on a separate validation set $\mathcal{M}_{\text{val}}$ with $100$, and $50$ samples held out for testing final performance metrics.


An intuitive baseline is to flatten the entire ResNet18 weight tensor $\mathbf{W}_m$ into a single high-dimensional vector and train an autoencoder directly on it. However, ResNet18 contains over 11 million parameters, so even a single linear layer mapping this input to a modest hidden dimension quickly becomes impractical. For instance, mapping the flattened weights ($\sim 11.7$M parameters) to a hidden dimension of 512 requires approximately 5.99 billion parameters. At 16-bit precision, this alone consumes nearly 12~GB of memory, demonstrating that a naive fully-connected approach is infeasible.

Alternative methods such as sequential tokenisation of layers (discussed in Section~\ref{sec:autoencoders}) provide a more scalable representation but introduce additional architectural complexity. Instead, we adopt a simpler yet effective two-stage compression approach, beginning with a \textit{Per-Named-Parameter PCA} transformation.  

In the first stage, dimensionality reduction is applied separately to each named parameter tensor (for example, \texttt{conv1.weight} or \texttt{fc.bias}) across all models. This preserves the modular structure of the network while tailoring the compression to the statistical properties of each parameter type. The process for determining which compression mode to apply is outlined in Algorithm~\ref{alg:pca-mode-selection}.

\begin{algorithm}[H]
\caption{Mode Selection During PCA Fitting}
\label{alg:pca-mode-selection}
\begin{algorithmic}[1]
\For{each named parameter in model zoo}
    \State $V \gets \text{calculate\_variance}(\text{parameter values across all models})$
    \State $D \gets \text{calculate\_dimension}(\text{parameter values across all models})$
    \If{$V < \text{VARIANCE\_THRESHOLD}$}
        \State \text{store\_only\_mean()} \Comment{Retain only mean}
    \ElsIf{$D \leq \text{DIMENSION\_LIMIT}$}
        \State \text{store\_centered\_weights()} \Comment{Retain centered weights}
    \Else
        \State \text{fit\_PCA()} \Comment{Fit PCA}
    \EndIf
\EndFor
\end{algorithmic}
\end{algorithm}

Three compression modes are used:
\begin{itemize}
    \item For high-variance, high-dimensional tensors, \textit{IncrementalPCA} projects the centred weights onto a small number of principal components, retaining maximal variance in a compact coefficient representation.
    \item For small tensors (four dimensions or fewer), only centering is applied, as PCA provides negligible benefit.
    \item For low-variance tensors, only the mean value is stored, discarding coefficients entirely for maximal compression.
\end{itemize}

Following PCA, the resulting coefficient vectors obtained for each named parameter are concatenated to form a single reduced representation for each model. Each vector contains the retained PCA coefficients for one named parameter, where the number of retained components \(k_i\) varies depending on that parameter's variance structure. The concatenated coefficient vector, denoted \(\mathbf{C}_m\), thus compactly represents all named parameters of model \(m\) in a unified format. This vector is normalised using dataset-wide statistics and then serves as input to the second compression step, a deep autoencoder. This two-stage compression process is illustrated in Figure~\ref{fig:pca_ae}, showing how PCA is applied per named parameter followed by a deep autoencoder producing a compact latent representation.


\begin{figure}[!t]
    \centering
    \includegraphics[width=0.9\linewidth]{pca_ae.png}
    \caption{Two-stage compression of model weights: PCA reduces each parameter tensor, followed by an autoencoder producing a compact latent representation.}

    \label{fig:pca_ae}
\end{figure}

% You never describe what theta_e or theta_d is
In the second stage, the autoencoder learns a lower-dimensional latent representation of the PCA-compressed vectors. The encoder maps the normalised coefficient vector $\mathbf{C}_m$ to a compact latent representation $\mathbf{z}_m \in \mathbb{R}^{D_{\text{latent}}}$:
\[
\mathbf{z}_m = f_{\text{enc}}(\mathbf{C}_m; \theta_e),
\]
where $\theta_e$ denotes the learnable parameters of the encoder. The decoder reconstructs the original coefficients as
\[
\hat{\mathbf{C}}_m = f_{\text{dec}}(\mathbf{z}_m; \theta_d).
\]
with $\theta_d$ representing the learnable parameters of the decoder. 
% Won't go that far, will more indicate that it minisises the **reconstruction loss** of the PCA vector.


Training minimises the reconstruction loss of the PCA-compressed model weights, extending the standard mean-squared error loss in Equation~\ref{eq:weight_recon_loss}:

\begin{equation}
  \mathcal{L}_{\text{AE}} = \frac{1}{|\mathcal{M}_{\text{train}}|} 
\sum_{m \in \mathcal{M}_{\text{train}}} 
\| \mathbf{C}_m - f_{\text{dec}}(f_{\text{enc}}(\mathbf{C}_m)) \|^2_2,
  \label{eq:pca_ae_loss_simplified}
\end{equation}

where $\mathbf{C}_m$ denotes the PCA coefficient vector representing all weights of model $m$. This formulation encourages the autoencoder to capture the overall structure of each model in a compact latent space while preserving the key variance across the full set of parameters.


After reconstruction, each parameter’s coefficients $\hat{\mathbf{c}}_{i,m}$ are transformed back into their original tensor shapes using the stored PCA statistics from the fitting stage. Specifically, for each named parameter $i$, the inverse PCA transform restores the weight tensor by reprojecting the coefficients into the original feature space and re-adding the mean vector:
\[
\hat{\mathbf{W}}_{i,m} = \mathbf{V}_i \hat{\mathbf{c}}_{i,m} + \boldsymbol{\mu}_i,
\]
where $\mathbf{V}_i$ is the PCA component matrix and $\boldsymbol{\mu}_i$ is the mean of the original parameter distribution. If PCA was not applied (for small or low-variance tensors), the stored centring or mean operations are inverted accordingly. 

The combination of PCA and an autoencoder provides several advantages. Applying PCA first drastically reduces the dimensionality of each model's weight vector, lowering both computational and memory demands for training the autoencoder. This compression enables faster and more efficient training while still capturing the dominant variance patterns present in the model weights.

Despite these benefits, the approach has inherent limitations. PCA is a linear technique, which means it can only approximate the weight space along linear directions and may miss non-linear dependencies between parameters. Additionally, it compresses based on the directions of variance observed in the training models, so models with weight configurations that vary along previously unrepresented directions may also be poorly encoded. Both factors can limit generalisation, particularly when the autoencoder encounters models with parameter distributions that differ substantially from the training set.

\subsection{Dataset Encoder}
\label{subsec:data_enc}

Following the discussion in Section~\ref{sec:contrastive} on contrastive learning and the CLIP framework, we adopt a pretrained CLIP visual encoder to extract semantic embeddings for the dataset classes. Each image $x$ is passed through the CLIP encoder $f_{\text{CLIP}}(\cdot)$, producing a 512-dimensional latent representation $z_x$:

\[
z_x = f_{\text{CLIP}}(x).
\]

To obtain a class-level representation, we compute the embeddings for all images belonging to a given class and average them, yielding a single 512-dimensional vector $\bar{z}_c$ that summarises the class:

\[
\bar{z}_c = \frac{1}{|X_c|} \sum_{x \in X_c} z_x,
\]

where $X_c$ denotes the set of images in class $c$. 

Each \(\bar{z}_{c_i}\) represents the embedding of a specific class in the dataset, with \(c_1, c_2, c_3\) consistently ordered according to the class indices in the classification layer. This consistent ordering preserves structure in the concatenated vector 

\[
\mathbf{z}_{\text{dataset}} = [\bar{z}_{c_1}; \bar{z}_{c_2}; \bar{z}_{c_3}] \in \mathbb{R}^{1536},
\]

which is then used as input to the shared encoding stage for alignment with the weight embeddings. By leveraging pretrained CLIP features, this approach provides a rich semantic summarisation of each class, ensuring the final dataset representation captures meaningful visual information relevant for model alignment.


\subsection{Results Encoder}
\label{subsec:results_enc}

To capture the performance of each model, we use the validation loss recorded at the checkpoint corresponding to the saved weights. Instead of using the raw scalar loss directly, we quantize the range of validation losses observed across the model zoo into discrete bins. Each bin is associated with a learnable 512-dimensional embedding vector, allowing the model to map a validation loss to a rich latent representation:

\[
\mathbf{r}_m = \text{Embedding}(\text{bin}(\mathcal{L}_{\text{val}, m})),
\]

where $\mathcal{L}_{\text{val}, m}$ is the validation loss of model $m$, and $\mathbf{r}_m \in \mathbb{R}^{512}$ is the corresponding learned embedding.  

This approach has several advantages. First, by representing the continuous range of validation losses with trainable embeddings, the system can learn a smooth latent space that captures nuanced performance differences between models. Second, it avoids the need to directly regress continuous loss values, which can be noisy and poorly scaled across diverse architectures or training schedules.

\section{Shared Encoding}
\label{sec:shared_enc}

The shared encoding stage aligns the dataset and results embeddings with the weight latent space. To achieve this, the dataset-level vector $\mathbf{z}_{\text{dataset}}$ from Section~\ref{subsec:data_enc} and the results embedding $\mathbf{z}_{\text{results}}$ from Section~\ref{subsec:results_enc} are first concatenated:

\[
\mathbf{z}_{\text{input}} = [\mathbf{z}_{\text{dataset}}; \mathbf{z}_{\text{results}}].
\]

This combined vector is then passed through a multi-layer perceptron (MLP) to produce a predicted embedding $\mathbf{z}_{\text{proj}}$ in the weight latent space:

\[
\mathbf{z}_{\text{proj}} = \text{MLP}(\mathbf{z}_{\text{input}}; \theta_{\text{MLP}}).
\]
s
Training minimises the NT-Xent contrastive loss, as described in Section~\ref{sec:contrastive} and defined in Equation~\ref{eq:multi_loss}, between $\hat{\mathbf{z}}_{\text{weight}}$ and the corresponding weight embedding $\mathbf{z}_{\text{weight}}$. Backpropagation is used to update the MLP parameters $\theta_{\text{MLP}}$:

\[
\mathcal{L}_{\text{NT-Xent}}(\mathbf{z}_{\text{proj}}, \mathbf{z}_{\text{weight}}) \rightarrow \min_{\theta_{\text{MLP}}}.
\]

This design ensures that the mapping from dataset and results space to the weight latent space is fully differentiable and preserves the reversibility of the weight representation. Importantly, the projection is performed from dataset and results to weight space rather than the reverse: applying a non-linear transformation directly to the weight embedding would break invertibility, making it impossible to reconstruct the original weight vector from the shared representation. By contrast, pushing the dataset and results embeddings to the weight space allows the latent weight vector to remain consistent and decodable via the trained autoencoder.

\begin{figure}[!t]
  \centering
  \includegraphics[width=0.918\linewidth]{conditional_model_sampling.png}
  \caption{Conditional model sampling pipeline shown, from dataset and results embedding, to a shared embedding space, to a reconstructed model}
  \label{fig:conditional}
\end{figure}

The overall conditional model sampling process is illustrated in Figure~\ref{fig:conditional}. Given a dataset embedding \(\mathbf{z}_{\text{dataset}}\) and results embedding \(\mathbf{z}_{\text{results}}\), the shared encoder projects their concatenated representation into the latent weight space as \(\mathbf{z}_{\text{proj}}\). Once aligned, this predicted latent vector can be passed through the trained weight decoder to reconstruct the full set of model parameters. In effect, this enables conditional generation of neural networks—sampling model weights that are consistent with a given dataset and desired performance profile.
 % include
\chapter{Results}
\label{chap:results}
% Report the measurement -> what does this mean? 
% Direct in how it is stated.

% Very passive voice in section 4, I think you need to relook at the writing and fix it, but overall a lot of passive voice is happening. Once you have added in the notes for section 4 (which I think is good), spend the time then cleaning this and making it just pop much better.

\section{Weight Autoencoder}
\label{sec:weights}
To evaluate the weight autoencoder, we examine the MSE during training. Figure~\ref{fig:weight_encoder_performance} plots the MSE on both the training and validation sets following PCA projection of the model weights. The training error decreases steadily over epochs and reaches a plateau, while the validation error follows a similar trajectory. This behaviour confirms that the model does not overfit the training data and generalises to unseen weight representations.

The MSE demonstrates that the weight autoencoder learns a meaningful mapping from weights to latent space and back. However, since the magnitude of the error depends on the scale of the input model weights, the absolute MSE value is not directly interpretable in terms of downstream model performance. Consequently, additional comparative evaluations are necessary to assess how well the learned representations preserve functionally relevant properties.

Figure \ref{fig:output_comparison} presents the results from a comparative experiment: unseen models are encoded and reconstructed, and the reconstructed models are then utilized to classify a specific test dataset. The output of the original model is compared against the output of the reconstructed model. Throughout the training, the model agreement on synthetic images was $100\%$. model agreement is defined as the percentage of test samples for which the classified label is identical between the original and reconstructed models. Correlation and cosine similarity were calculated on the pre-softmax outputs. It is evident that the weight autoencoder is capable of learning model representations with sufficient fidelity to reconstruct models exhibiting highly correlated outputs---specifically, $0.95+\%$ correlation---compared to the original model's outputs.

The same model output comparison procedure is then executed, but using the data upon which the original model was initially trained. Table \ref{tab:weight_real_data} summarizes these cross-dataset performance metrics.

\begin{figure*}[!t]
    \centering
    \subfloat[Training and Validation Loss]{
    \includegraphics[width=0.48\textwidth]{weight_encoder/combined_loss_plot.png}
        \label{fig:weight_encoder_performance}
    }
    \subfloat[Test Model Output Comparison]{
    \includegraphics[width=0.48\textwidth]{weight_encoder/output_comparison.png}
        \label{fig:output_comparison}
    }
    \caption{Performance curves of the weight autoencoder, showing loss progression and comparative output metrics.}
    \label{fig:combined_plots}
\end{figure*}



\begin{table}[!h]
    \centering
    \caption{Model Reconstruction Performance on Training Data}
    \label{tab:weight_real_data}
    \begin{tabularx}{0.8\linewidth}{@{}lcc@{}} 
        \toprule
        \textbf{Metric} & \textbf{Mean Value} & \textbf{Range} \\
        \midrule
        Cosine Similarity (Pre-Softmax Output) & $-0.082$ & $[-0.547,\, 0.526]$ \\
        Correlation (Pre-Softmax Output) & $0.016$ & $[-0.329,\, 0.344]$ \\
        Prediction Agreement & $31.21\%$ & $[2.06\%,\, 64.88\%]$ \\
        \bottomrule
    \end{tabularx}
\end{table}

These results indicate that the weight autoencoder is unable to reconstruct models with sufficient fidelity to preserve their original decision boundaries. The sharp decline in output correlation between the reconstructed and real models shosws that, although the autoencoder captures a coarse structural representation of the ResNet18 weights, critical fine-grained information necessary for accurate functional behaviour is lost.

This degradation originates from the initial PCA compression stage. While PCA retains the principal directions of variance and thereby the most dominant components of the data, it remains a linear approximation of a fundamentally non-linear parameter space. As a result, essential non-linear dependencies within the weight configurations may have been discarded during projection. Given that the autoencoder itself achieves a low reconstruction loss and demonstrates good generalisation, as shown in Figure~\ref{fig:weight_encoder_performance}, the poor reconstruction performance can therefore be reasonably attributed to information loss introduced by PCA.

\begin{figure}[!t]
  \centering
  \includegraphics[width=0.6\linewidth]{weight_encoder/hidden_dim_loss.png}
  \caption{Validation NT-Xent loss of the weight autoencoder for a range of hidden dimensions. Each list represents a layer configuration, with numbers indicating layer sizes.}
  \label{fig:hidden_dim_loss}
\end{figure}

Further evidence of this information loss is shown in Figure~\ref{fig:hidden_dim_loss}, which presents the validation loss across four weight autoencoders with varying architectures. The simplest configuration ('[]') exhibits the most stable and gradually decreasing validation loss, indicating stronger generalisation. In contrast, as model capacity increases—from '[512]' to '[2048,1024,512]'—the validation loss rises and signs of overfitting become apparent. This behaviour suggests that the autoencoder receives an insufficiently informative learning signal, likely due to the high complexity and non-linearity of the underlying weight distribution. Consequently, the overfitting observed in larger models further supports the hypothesis that the PCA compression stage removed too much essential information for accurate reconstruction.

\section{Shared Encoder}
\label{sec:shared}
\begin{figure}[!t]
    \centering
    \includegraphics[width=0.65\linewidth]{shared/nt_xent_loss_plot.png}
    \caption[Shared encoder training and validation losses.]{Training and validation losses for the shared encoder. The validation loss remains high, indicating a lack of generalisable signal due to information loss in the weight embeddings.}
    \label{fig:shared_loss}
\end{figure}
The shared encoder was trained following the procedure described in Section~\ref{sec:shared_enc}, with both training and validation losses monitored throughout. Figure~\ref{fig:shared_loss} shows that the training loss decreases rapidly at first and then decreases at a steady rate, while the validation loss decreases slightly before stabilising at a relatively high value of \(\sim\)3.1.

This behaviour suggests that the shared encoder is successfully fitting the training data but fails to generalise to unseen data. The persistently high validation loss indicates that the model cannot extract a meaningful or consistent signal from the validation set, likely due to information loss introduced earlier in the weight encoding process. As the weight representations lack sufficient structure or variance relevant to the corresponding $(\mathcal{D}, R)$ pairs, the shared encoder effectively trains on noise. 

Interestingly, the validation loss does not worsen over time, which would typically occur during overfitting. This further supports the hypothesis that there is little to no informative signal to learn—changes to the shared encoder's weights have minimal effect on generalisation performance, implying that the training dynamics are dominated by random correlations rather than meaningful alignment.

\begin{figure}[!t]
    \centering
    \makebox[\textwidth][c]{\hspace{-1cm}\includegraphics[width=1.1\linewidth]{heat_map.png}}
    \caption[Cosine similarity heat map for weight and projected embeddings.]{Cosine similarity heat map between five weight embeddings and their corresponding projected embeddings $z_{\text{proj}}$.}
    \label{fig:heat_map}
\end{figure}

Figure~\ref{fig:heat_map} provides further insight into this behaviour by visualising the cosine similarity between five weight embeddings and their corresponding projected embeddings ${z}_{\text{proj}}$ within the training set. The x-axis represents $(\mathcal{D}, R)$ pairs sampled from both early and late training epochs, while the y-axis shows the corresponding latent weight vectors produced by the autoencoder. The numbers in parentheses (for example, “(41)”) indicate the training epoch at which a specific dataset--result pair or weight embedding was generated. Darker regions in the heatmap correspond to higher cosine similarity, meaning that the reconstructed embedding more closely matches the original latent representation, while lighter regions indicate lower alignment. This allows the figure to highlight how reconstruction fidelity changes across training progress and between different dataset--result pairs.

High cosine similarity values (red regions) are observed along the diagonal, showing that matching pairs of weights and $(\mathcal{D}, R)$ embeddings are aligned in the shared space. Negative pairs, shown as white or blue, exhibit low similarity as expected. Notably, the same $(\mathcal{D}, R)$ pairs across different epochs also display higher similarity, despite not being explicitly defined as positives. This shows that the shared encoder partially retains the relational structure among triplets $(\mathcal{D}, R, W)$ across training stages, even in the absence of a strong global signal.

 % include
\graphicspath{{conclusion/fig/}}

\chapter{Summary}
\label{chap:conclusion} % include

% \section{Section heading}

% This is some section with two table in it: Table~\ref{tbl:exemplars} and Table~\ref{tbl:abx_speaker}.

% \begin{table}[!h]
%     \mytable
%     \caption{Performance of the unconstrained segmental Bayesian model on TIDigits1 over iterations in which the reference set is refined.}
%     \begin{tabularx}{\linewidth}{@{}lCCCCC@{}}
%         \toprule
%         Metric     & 1 & 2 & 3 & 4 & 5 \\
%         \midrule
%         WER (\%)                        & $35.4$ & $23.5$ & $21.5$ & $21.2$ & $22.9$ \\
%         Average cluster purity (\%)       & $86.5$ & $89.7$ & $89.2$ & $88.5$ & $86.6$ \\
%         Word boundary $F$-score (\%)         & $70.6$ & $72.2$ & $71.8$ & $70.9$ & $69.4$ \\
%         Clusters covering 90\% of data   & 20             & 13 & 13 & 13 & 13 \\
%         \bottomrule
%     \end{tabularx}
%     \label{tbl:exemplars}
% \end{table}


% \begin{table}[!h]
%     \renewcommand{\arraystretch}{1.1}
%     \centering
%     \caption{A table with an example of using multiple columns.}
%     \begin{tabularx}{0.65\linewidth}{@{}lCCr@{}}
%         \toprule
%         & \multicolumn{2}{c}{Accuracy (\%)} \\
%         \cmidrule(lr){2-3}
%         Model    & Intermediate & Output & Bitrate\\
%         \midrule
%         Baseline & 27.5         & 26.4   & 116 \\
%         VQ-VAE   & 26.0         & 22.1   & 190 \\
%         CatVAE   & 28.7         & 24.3   & 215 \\
%         \bottomrule
%     \end{tabularx}
%     \label{tbl:abx_speaker}
% \end{table}

% \newpage

% This is a new page, showing what the page headings looks like, and showing how to refer to a figure like Figure~\ref{fig:cae_siamese}.

% \begin{figure}[!t]
%     \centering
% %     \includegraphics[width=\linewidth]{cae_siamese}
%     \includegraphics[width=0.918\linewidth]{cae_siamese}
%     \caption[I am the short caption that appears in the list of figures, without references.]{
%     (a) The cAE as used in this chapter. The encoding layer (blue) is chosen based on performance on a development set.
%     (b) The cAE with symmetrical tied weights. The encoding from the middle layer (blue) is always used.
%     (c) The siamese DNN. The cosine distance between aligned frames (green and red) is either minimized or maximized depending on whether the frames belong to the same (discovered) word or not.
%     A cAE can be seen as a type of DNN~\cite{dahl+etal_taslp12}.
%     }
%     \label{fig:cae_siamese}
% \end{figure}


% The following is an example of an equation:
% \begin{equation}
% P(\vec{z} | \vec{\alpha}) = \int_{\vec{\pi}} P(\vec{z} | \vec{\pi}) \, p(\vec{\pi} | \vec{\alpha}) \, \textrm{d} \vec{\pi}
% = \int_{\vec{\pi}} \prod_{k = 1}^K \pi_k^{N_k} \frac{1}{B(\vec{\alpha})} \prod_{k = 1}^K \pi_k^{\alpha_k - 1} \, \textrm{d} \vec{\pi}
% \label{eq:example_equation}
% \end{equation}
% which you can subsequently refer to as~\eqref{eq:example_equation} or Equation~\ref{eq:example_equation}.
% But make sure to consistently use the one or the other (and not mix the two ways of referring to equations).





% Bibliography
\bibliography{mybib} % include

% End matter
% \appendix
% \input{appendices/appendices}

\end{document}

